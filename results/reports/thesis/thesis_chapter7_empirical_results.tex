\chapter{ 7: Empirical Results

\section{7.1. Data & Sample

This section provides a brief recap of the data and methodology used in the empirical analysis.

#\section{Universe

The analysis covers \textbf{7 European markets\textbf{: Germany (DE), France (FR), Italy (IT), Spain (ES), Sweden (SE), United Kingdom (UK), and Switzerland (CH). After data cleaning and quality filters, the final sample consists of \textbf{219 stocks\textbf{ with complete return data.

#\section{Period

The analysis period spans \textbf{2020-12-01 to 2025-12-01\textbf{, using \textbf{monthly\textbf{ return frequency. This provides \textbf{59 months of returns\textbf{ (60 months of prices) for each stock, covering a period that includes:
- Post-COVID recovery (2021)
- Interest rate hikes and inflation concerns (2022)
- Economic recovery and policy normalization (2023-2025)

#\section{Market Proxies

For each country, beta is estimated against the \textbf{MSCI country index\textbf{ (accessed via iShares ETFs), not the local stock exchange index. This choice is motivated by:
- Broader market coverage (includes mid-caps)
- Consistent methodology across countries
- Empirically superior explanatory power

The market proxies used are:
- Germany: MSCI Germany (via EWG)
- France: MSCI France (via EWQ)
- Italy: MSCI Italy (via EWI)
- Spain: MSCI Spain (via EWP)
- Sweden: MSCI Sweden (via EWD)
- United Kingdom: MSCI UK (via EWU)
- Switzerland: MSCI Switzerland (via EWL)

#\section{Risk-Free Rate

The risk-free rate is the \textbf{3-month government bond yield\textbf{ for each country:
- \textbf{Eurozone countries\textbf{ (Germany, France, Italy, Spain): German 3-month Bund (EUR)
- \textbf{Sweden\textbf{: Swedish 3-month government bond (SEK)
- \textbf{United Kingdom\textbf{: UK 3-month Treasury Bill (GBP)
- \textbf{Switzerland\textbf{: Swiss 3-month government bond (CHF)

All rates are converted from annual percentage to monthly percentage for consistency with monthly return frequency.

#\section{Excess Returns

All returns are calculated as \textbf{simple (arithmetic) percentages\textbf{, not logarithmic returns. The excess return calculation follows:

\textbf{Stock excess return:\textbf{
$$E_{i,t} = R_{i,t} - R_{f,t}$$

\textbf{Market excess return:\textbf{
$$E_{m,t} = R_{m,t} - R_{f,t}$$

Where $R_{i,t}$ is the simple monthly return of stock $i$ in month $t$, $R_{m,t}$ is the MSCI country index return, and $R_{f,t}$ is the risk-free rate.

See Table~\ref{tab:6 for a summary of descriptive statistics by country.

---

\section{7.2. Time-Series CAPM Results

This section addresses the question: \textbf{"Does the market explain each individual stock over time?"\textbf{

#\section{Method

For each stock $i$, we estimate the time-series CAPM regression:

$$R_{i,t} - R_{f,t} = \alpha_i + \beta_i (R_{m,t} - R_{f,t}) + \varepsilon_{i,t}$$

where:
- $R_{i,t} - R_{f,t}$ is the excess return of stock $i$ in month $t$
- $R_{m,t} - R_{f,t}$ is the excess return of the market (MSCI country index) in month $t$
- $\beta_i$ is the market beta (sensitivity to market movements)
- $\alpha_i$ is the intercept (abnormal return)
- $\varepsilon_{i,t}$ is the error term

\textbf{Estimation:\textbf{ Each regression uses \textbf{59 monthly observations\textbf{ (January 2021 to November 2025), estimated via Ordinary Least Squares (OLS).

#\section{Key Aggregate Statistics

The time-series regressions yield the following aggregate results:

- \textbf{Valid stocks:\textbf{ 219 (out of 249 total)
- \textbf{Average beta:\textbf{ 0.688
- \textbf{Median beta:\textbf{ 0.646
- \textbf{Average R²:\textbf{ 0.235
- \textbf{Average alpha:\textbf{ 0.144%

#\section{Results by Country

Table~\ref{tab:1 presents the CAPM time-series summary by country. Key findings:

- \textbf{Beta values\textbf{ are economically reasonable, with median betas ranging from approximately 0.53 to 0.67 across countries. Most stocks have betas below 1.0, as expected for many European large-cap stocks that are less volatile than their respective market indices.

- \textbf{R² values\textbf{ average approximately 0.24, indicating that the market factor explains approximately 23.5% of return variation on average. This represents a \textbf{non-trivial but limited\textbf{ explanatory power.

- \textbf{Significant betas\textbf{ (p < 0.05) represent approximately 95.9% of the sample, indicating that for most stocks, beta is statistically distinguishable from zero.

#\section{Distribution of Betas and R²

Figure~\ref{fig:1 shows the distribution of betas by country. The distributions are centered around 0.6-0.7, with relatively few extreme values (after removing outliers). Figure~\ref{fig:2 shows the distribution of R² values, confirming that while CAPM explains a meaningful portion of return variation, a substantial fraction (approximately 0.76) remains unexplained by market risk alone.

#\section{Interpretation

\textbf{In time series, the CAPM is partially useful:\textbf{ The market factor explains some risk, but not all. The average R² of 0.235 suggests that:

1. \textbf{Market risk matters:\textbf{ Beta captures a meaningful portion of stock return variation, confirming that market movements are an important driver of individual stock returns.

2. \textbf{Idiosyncratic risk dominates:\textbf{ Approximately 0.76 of return variation is unexplained by the market, indicating that firm-specific factors, sector effects, and other non-market drivers play a substantial role.

3. \textbf{Betas are plausible:\textbf{ The median beta of 0.646 is consistent with expectations for developed European markets, where large-cap stocks typically exhibit moderate market sensitivity.

4. \textbf{Positive alphas:\textbf{ The average alpha of 0.144% suggests that stocks, on average, earn returns slightly higher than what CAPM predicts based on their beta alone.

---

\section{7.3. Cross-Sectional Pricing Test – Fama-MacBeth

This section addresses the \textbf{core CAPM question\textbf{: \textbf{"Do higher betas earn higher average returns?"\textbf{

#\section{Method

We employ the \textbf{Fama-MacBeth (1973) two-pass regression\textbf{ methodology:

\textbf{Pass 1 (Time-Series):\textbf{ Estimate $\beta_i$ for each stock $i$ from time-series regressions (as in Section 7.2).

\textbf{Pass 2 (Cross-Sectional):\textbf{ For each month $t$, run a cross-sectional regression:

$$R_{i,t} = \gamma_{0,t} + \gamma_{1,t} \beta_i + u_{i,t}$$

where:
- $R_{i,t}$ is the return of stock $i$ in month $t$
- $\beta_i$ is the beta estimated in Pass 1
- $\gamma_{1,t}$ is the \textbf{market price of risk\textbf{ in month $t$ (the reward for bearing beta risk)
- $\gamma_{0,t}$ is the intercept in month $t$ (the return on a zero-beta portfolio)

We then \textbf{average\textbf{ $\gamma_0$ and $\gamma_1$ across all 59 months and compute Fama-MacBeth standard errors and t-statistics.

#\section{Core Results

Table~\ref{tab:2 presents the Fama-MacBeth test results. The key findings are:

- \textbf{Average $\gamma_1$ (market price of risk):\textbf{ -0.9662
  - \textbf{t-statistic:\textbf{ -1.199
  - \textbf{p-value:\textbf{ 0.2356
  - \textbf{Interpretation:\textbf{ NOT statistically significant at conventional levels (5% or 10%)

- \textbf{Average $\gamma_0$ (intercept):\textbf{ 1.5385%
  - \textbf{t-statistic:\textbf{ 4.444
  - \textbf{p-value:\textbf{ 0.000040
  - \textbf{Interpretation:\textbf{ HIGHLY statistically significant (p < 0.0001)

#\section{Interpretation

\textbf{CAPM Prediction:\textbf{
- $\gamma_1 > 0$ and statistically significant → Higher beta stocks should earn higher returns
- $\gamma_0 \approx 0$ (if risk-free rate is correctly specified) → No abnormal return for zero-beta assets

\textbf{Our Results:\textbf{
- \textbf{$\gamma_1$ is negative and insignificant\textbf{ → Beta is \textbf{not rewarded\textbf{ in the cross-section. Higher beta stocks do not earn higher average returns than lower beta stocks.

- \textbf{$\gamma_0$ is strongly positive and significant\textbf{ → Even a zero-beta asset earns a positive excess return (approximately 1.54% per month), suggesting that the zero-beta rate exceeds the risk-free rate.

\textbf{Conclusion:\textbf{ This is a \textbf{rejection of the CAPM in the cross-section\textbf{, despite time-series betas being sensible and statistically significant. The model fails its main prediction: that beta should explain cross-sectional variation in expected returns.

Figure~\ref{fig:3 shows the time series of $\gamma_{1,t}$ (monthly market price of risk). The values bounce around zero with no clear positive trend, providing visual confirmation that beta is not consistently priced across months.

---

\section{7.4. Robustness & Extensions

This section demonstrates that the CAPM rejection is robust across different specifications, subperiods, and sample selections.

#\section{7.4.1. Subperiod Tests

To examine whether the results are driven by specific market conditions, we split the sample into two subperiods:

- \textbf{Period A:\textbf{ January 2021 – December 2022 (24 months)
- \textbf{Period B:\textbf{ January 2023 – November 2025 (35 months)

Table~\ref{tab:3 presents the Fama-MacBeth results for each subperiod.

\textbf{Period A (2021-2022):\textbf{
- $\gamma_1 = -1.7935$ (t = -1.361, not significant)
- This period includes post-COVID reopening volatility and the initial phase of interest rate hikes.

\textbf{Period B (2023-2025):\textbf{
- $\gamma_1 = -0.3989$ (t = -0.391, not significant)
- This period includes economic recovery and policy normalization.

\textbf{Interpretation:\textbf{ In \textbf{both subperiods\textbf{, beta is not priced. The CAPM rejection is \textbf{not driven by just one abnormal year\textbf{ but appears to be a persistent feature of the data across different market regimes.

#\section{7.4.2. Country-Level Fama-MacBeth

Table~\ref{tab:4 presents Fama-MacBeth results for each country separately. Key findings:

- \textbf{France:\textbf{ $\gamma_1 = -2.4455$ (t = -1.962) — borderline significant \textbf{negative\textbf{ pricing
- \textbf{Sweden:\textbf{ $\gamma_1 = -2.4246$ (t = -1.881) — also \textbf{negative\textbf{
- \textbf{Other countries:\textbf{ Show weak or insignificant relationships

\textbf{Interpretation:\textbf{ In several countries (notably France and Sweden), \textbf{higher beta stocks tend to earn lower average returns\textbf{, the exact opposite of what CAPM predicts. This suggests that:
- Local factors, sector weights, or risk-aversion patterns may dominate beta effects
- Market efficiency varies across European markets
- Country-specific institutional or behavioral factors may be at play

#\section{7.4.3. Beta-Sorted Portfolios

This analysis provides \textbf{the strongest visual evidence\textbf{ of CAPM failure.

We sort all stocks into \textbf{5 portfolios\textbf{ by their estimated beta (P1 = lowest beta, P5 = highest beta) and compute equal-weighted monthly returns for each portfolio.

\textbf{Results (Table~\ref{tab:5):\textbf{
- Portfolio 1 (lowest beta): Average return = 1.20%, Portfolio beta = 0.360
- Portfolio 5 (highest beta): Average return = 0.54%, Portfolio beta = 1.102

\textbf{Key Finding:\textbf{ Higher beta portfolios have \textbf{lower\textbf{ average returns, creating a \textbf{negative slope\textbf{ in the beta-return relationship.

\textbf{Figure~\ref{fig:5\textbf{ plots portfolio beta against average return. Under CAPM, we would expect a clear \textbf{upward-sloping line\textbf{. Instead, the relationship is \textbf{flat to slightly negative\textbf{, providing intuitive visual confirmation that beta does not explain cross-sectional return variation.

This is \textbf{extremely strong evidence\textbf{ that supports the Fama-MacBeth rejection of CAPM.

#\section{7.4.4. Clean Sample Test

We remove outliers to ensure results are not driven by a few anomalous stocks:
- Extreme betas (|$\beta$| > 5)
- Very low R² (R² < 0.05)
- Insignificant betas (p-value > 0.10)

After cleaning, we re-run the Fama-MacBeth test. The results confirm the main finding: $\gamma_1$ remains negative and insignificant, demonstrating that the CAPM rejection is \textbf{robust to sample cleaning\textbf{.

---

\section{7.5. Economic Interpretation & Link to Literature

#\section{Why Might CAPM Fail 2021-2025 in Europe?

Several factors may explain the CAPM failure during this period:

\textbf{1. Macro Regime Shifts:\textbf{
- \textbf{COVID recovery (2021):\textbf{ Unprecedented fiscal and monetary stimulus created sector-specific winners and losers that were not captured by market beta alone.
- \textbf{Inflation spike (2022):\textbf{ Rapid interest rate hikes by ECB, BoE, SNB, and Riksbank created winners (banks, value stocks) and losers (growth stocks, long-duration assets) based on factors beyond beta.
- \textbf{Policy normalization (2023-2025):\textbf{ Transition periods often see factor rotations (value vs. growth, cyclical vs. defensive) that beta cannot capture.

\textbf{2. Sector Tilts in European Indices:\textbf{
European markets have heavy concentrations in:
- \textbf{Banks and financials\textbf{ (sensitive to interest rates, not just market movements)
- \textbf{Cyclical industries\textbf{ (automotive, industrials) with earnings cycles driven by macro factors
- \textbf{Defensive sectors\textbf{ (utilities, consumer staples) that may outperform during volatility

These sector effects create return patterns that are orthogonal to market beta.

\textbf{3. Alternative Drivers of Returns:\textbf{

Instead of beta, returns may be driven by:

- \textbf{Size factor:\textbf{ Small stocks may outperform large stocks (size premium)
- \textbf{Value factor:\textbf{ Value stocks may outperform growth stocks (value premium)
- \textbf{Quality factor:\textbf{ High-profitability, low-investment stocks may outperform
- \textbf{Momentum factor:\textbf{ Recent winners may continue to outperform
- \textbf{Sector/specific risk:\textbf{ Energy, tech, luxury, financials have distinct risk-return profiles
- \textbf{ESG and flows:\textbf{ ESG considerations and ETF flows may create return patterns unrelated to beta

#\section{Link to Classic Findings

Our results are \textbf{consistent with\textbf{ the seminal work of Fama & French (1992, 1993):

- \textbf{Fama & French (1992):\textbf{ Found that beta does not explain cross-sectional variation in returns; size and book-to-market (value) do.
- \textbf{Fama & French (1993):\textbf{ Proposed a three-factor model (market, size, value) that outperforms CAPM.

Our finding that $\gamma_1$ is insignificant aligns with their conclusion that \textbf{beta alone is insufficient\textbf{ to explain expected returns.

More recent literature finds \textbf{mixed evidence\textbf{ for CAPM, especially in:
- Short samples and turbulent periods (like 2021-2025)
- International markets where local factors matter
- Periods of monetary policy shifts

#\section{Implications for Equity Analysis

\textbf{As an equity analyst, relying only on $\beta$ as a risk measure is insufficient.\textbf{ Our results suggest that:

1. \textbf{Multi-factor models are essential:\textbf{ Consider exposures to:
   - Value vs. growth
   - Size (small vs. large)
   - Quality (profitability, investment)
   - Momentum
   - Sector factors

2. \textbf{Sector risk matters:\textbf{ Beyond market beta, sector-specific risks (e.g., interest rate sensitivity for banks, commodity exposure for energy) are crucial.

3. \textbf{Balance sheet strength:\textbf{ Financial leverage, liquidity, and credit quality create return patterns not captured by beta.

4. \textbf{Macro sensitivity:\textbf{ Understanding how stocks respond to inflation, interest rates, and currency movements is more important than market beta alone.

5. \textbf{CAPM as a benchmark:\textbf{ While CAPM fails empirically, it remains useful as a \textbf{starting point\textbf{ for cost-of-equity estimation, but should be adjusted for:
   - Size premiums
   - Industry-specific risk factors
   - Company-specific considerations

#\section{Conclusion

This analysis provides \textbf{robust evidence\textbf{ that the Capital Asset Pricing Model fails to explain cross-sectional variation in European stock returns during 2021-2025. The results are consistent across:
- Time-series regressions (moderate explanatory power, R² ≈ 0.24)
- Cross-sectional tests (beta not priced, $\gamma_1$ insignificant)
- Robustness checks (results hold across subperiods, countries, and sample specifications)

These findings support the use of \textbf{multi-factor models\textbf{ (Fama-French, Carhart) that incorporate size, value, profitability, investment, and momentum factors beyond market beta alone.

---

\section{References

Fama, E. F., & French, K. R. (1992). The cross-section of expected stock returns. *Journal of Finance*, 47(2), 427-465.

Fama, E. F., & French, K. R. (1993). Common risk factors in the returns on stocks and bonds. *Journal of Financial Economics*, 33(1), 3-56.

Fama, E. F., & MacBeth, J. D. (1973). Risk, return, and equilibrium: Empirical tests. *Journal of Political Economy*, 81(3), 607-636.

---

\textbf{End of Chapter 7\textbf{
